\documentclass[11pt]{article}
\usepackage{geometry}                % See geometry.pdf to learn the layout
\usepackage{graphicx}
\usepackage{amssymb}
\usepackage{epstopdf}

\title{Compte-rendu du TEA "Le compte est bon"}
\author{\textsc{Valentin GAUTHIER}\\ Alexandre TORRES--LEGUET}

\begin{document}
\maketitle

\section{Introduction}
%présentation du problème, le contexte

\section{Développement}
%organisation du programme
%organisation du groupe (qui a fait quoi)
%difficultés

\section{Résultats}
%présentation des résultats, jeux d'essais, analyse, temps d'exec...

\section{Conclusion}
%analyser les problèmes, perspectives, bibliographie


\end{document}  

%Choisir la manière de représenter les cartons restant à utiliser dans les appels à la fonction de parcours à la volée en profondeur d’abord
%Ecrire le prototype de cette fonction 
%Ecrire l’algorithme permettant d’implémenter cette fonction  
%Sur le schéma précédent, déterminer dans quel ordre les noeuds de l’arbre seront parcourus par votre algorithme en ajoutant pour chacun son indice dans le parcours

%attention petite erreur sur le schema (noeud 14 a peu pres, + doit etre *)
